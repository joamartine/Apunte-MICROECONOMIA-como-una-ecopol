\chapter{Anexo}

\section{Cournot con restricciones de capacidad}

De forma complementaria se puede levantar el supuesto de que una sola empresa podría servir a todo el mercado si así lo quisiera. Algo más parecido a la realidad es que una sola firma no puede servir a todo el mercado aunque así lo quisiera. La capacidad de una firma puede denotarse como $K$. Frente a una demanda inversa lineal $P(Q) = A-Q$ y considerando dos firmas idénticas de $c=0$ tendremos restricciones activas de capacidad en caso de que $K \leq A$. 

En caso de que compitan la pregunta es qué parte de la curva de demanda sirve cada firma. Se puede asumir racionamiento eficiente, en caso de haber dos precios distintos el más barato será el primero en venderse a los que están más dispuestos a pagar. Una vez vendida toda la capacidad de la firma más barata la otra firma se queda con el sobrante.