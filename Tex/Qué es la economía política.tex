En 1878, el filósofo y economista marxista Friedrich Engels publicó el segundo volumen la que es considerada como su mayor aporte a la teoría marxista de la época. Se titula \textit{Herrn Eugen Dühring's Umwälzung der Wissenschaft}, esto es ``La revolución científica del Señor Eugen Dühring'', un texto que se conoce como el ``Anti-Dühring''.\footnote{Engels, Friederich (1877/78) \textit{Herrn Eugen Dühring's Umwälzung der Wissenschaft. Philosophie, Politische Oekonomie, Socialismus}, Genossenschafts-Buchdruckerei, Leipzig, Imperio Alemán} En ése texto, escrito para refutar las tesis filosóficas y políticas de un rival de Karl Marx, quien se encontraba, en ese momento, demasiado ocupado trabajando en ``El capital.'' se ofrece la definición clásica de Economía Política: \\

\hfill\begin{minipage}{\dimexpr\textwidth-2cm}
``La economía política, en el sentido más amplio, es la ciencia que estudia las leyes que gobiernan la producción y el intercambio de los medios materiales de subsistencia en la sociedad humana... por lo tanto, la economía política es esencialmente una ciencia histórica. Se ocupa de material que es histórico y que está en constante cambio.''
\end{minipage}

\marginnote{\textbf{Economía política:} Definición}[0cm] % nota al margen, los cm del corchete es la distancia hacia abajo o hacia arriba de la nota con respecto a donde está escrito en el código. (2 cm hacia abajo [2cm], 2cm hacia arriba [-2cm])