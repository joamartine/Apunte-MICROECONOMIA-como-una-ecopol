\documentclass[12pt]{book}

\usepackage[toc]{appendix} % para crear ambientes de apéndice
\usepackage{amsmath,amssymb, amsthm} % Para mejorar ecuaciones
\usepackage[utf8]{inputenc}
\usepackage[spanish]{babel} % Español
\usepackage[nice]{nicefrac}  % Para mejorar fracciones
\usepackage{graphicx} % Para insertar graficos
\usepackage{float} % Para manejar la ubicacion de graficos
\usepackage[colorlinks=true, allcolors=black]{hyperref} % Para hipervincular las referencias dentro del texto
\usepackage[Conny]{fncychap} % Capítulos

\usepackage{marginnote}% Paquete para notas al márgen
    \setlength{\marginparwidth}{4cm} % Ancho de la nota al margen
    \setlength{\marginparsep}{0.5cm} % Distancia mínima entre la nota y el texto principal
\usepackage{multirow,array}


\title{MICROECONOMÍA\\como una economía política}
\author{Oscar Landerretche\\Universidad de Chile}

\begin{document}

Este documento es una colección de apuntes de materia elaborados para armar el libro del ramo de Economía Política por el profesor Óscar Landerretche en la Facultad de Economía y Negocios de la Universidad de Chile. \\

Esta es una versión muy prematura del apunte por lo que puede contener errores. La última actualización se hizo en marzo de 2024, cualquier error por favor notificarme al correo, joamartine@fen.uchile.cl.

\maketitle        



%%%%%%%%%%%%%%%%%%%%%%%%%%%%%%%%%%%%%%%%%%%%%%%%%%%%%%%%%%%%%%%%%%%%%%

\newpage 

\setcounter{tocdepth}{1} % La tabla de contenidos solo llegará a mostrar hasta section
\tableofcontents

\chapter{Introducción}

\section{¿Qué es la economía política?}

\input{Tex/Qué es la economía política}

\section{Microeconomía y macroeconomía}

\section{El problema de la racionalidad}

\chapter{Epistemología económica}

\section{El homo económico}

\section{Racionalidad limitada}

\section{Inferencia y comportamiento}

\section{Intuición y comportamiento}

\section{Trampas de ignorancia}

\chapter{El problema de lo privado y lo público}

\section{La economía descentralizada}

\subsection{Motivación del modelo de equilibrio general}

Los modelos de equilibrio parcial buscan describir los ajustes de oferta y demanda de un mercado específico tratando el resto de la economía como constante. Estos modelos de equilibrio parcial fueron desarrollados en gran parte por \textbf{Alfred Marshall}.\marginnote{\textbf{Alfred Marshall:} Economista británico que durante finales del siglo XIX traslapó conceptos de la economía clásica con la escuela marginalista.}[-2cm] Por ejemplo, si analizaramos el mercado de manzanas supondríamos que los costos de los factores de producción son exógenos, esto significaría que los costes de trabajo (salarios) se mantendrían constantes independiente de la dotación y equilibrio del modelo. 

Encontrándose el precio de mercado en equilibrio, en ninguna extensión se aborda el equilibrio del mercado laboral que hay detrás de la producción de manzanas. En la realidad como distintos mercados están interactuando constantemente un shock en un mercado puede derivar a múltiples ajustes en mercados relacionados. Por ejemplo, un subsidio al empleo bajaría los costes por trabajo, potencialmente aumentando la producción de manzanas. Estos fenómenos no se pueden analizar con modelos de equilibrio parcial pues se limitan a un sólo mercado dejando los factores como constantes dadas. 

Por otro lado, los modelos de \textbf{equilibrio general}\marginnote{\textbf{Equilibrio General:} Modelo que busca explicar el equilibrio dentro de una economía con más de un mercado en constante interacción.}[-1cm] describen dos o más mercados relacionados que se ajustan al mismo tiempo, lo cual nos podría dar una imagen más completa de los efectos de distintos shocks en la economía como un todo. Analizar el equilibrio de un número arbitrario de mercados es complejo, por suerte limitarnos a dos mercados y plantear diversos supuestos nos permite aun así, abstraer las conclusiones que encontramos relevantes para esta ocasión. 

\subsection{Planteamiento del modelo}

Una razón poderosa por la que limitarnos a analizar solo dos mercados es la \textbf{Ley de Walras}.\marginnote{\textbf{Ley de Walras:} En una economía de $n$ mercados, que $n-1$ de ellos esté en equilibrio implica que el otro también lo estará.}[-1cm] La ley de Walras es un principio desarrollado por \textbf{León Walras}\marginnote{\textbf{León Walras:} Economista francés de la escuela (neoclásica) de Lausana. Considerado fundador de la economía matemática.}[3cm] que plantea que al tener dos mercados, si el exceso de demanda de un bien es cero implica que el exceso de demanda del otro bien también será cero. Por lo tanto, los dos mercados están en equilibrio.\footnote{Con exceso de demanda igual a cero entiéndase como oferta igual a demanda.} Generalizando, si en una economía de $n$ mercados hay $n-1$ mercados en equilibrio, entonces el último mercado (el n-ésimo) también estará en equilibrio. Es por esto que limitarnos a analizar sólo dos mercados no influye mucho en las conclusiones relevantes del modelo. 

Vamos a describir una economía de intercambio entre dos individuos que al consumir dos bienes se formarán dos mercados. Estos individuos serán racionales, maximizarán su función de utilidad sujeto a una restricción presupuestaria.\footnote{Se asume que la función de utilidad es del tipo $u_i: \mathbb{R}_+^2 \xrightarrow{} \mathbb{R}$, la cual cumple con ser (i) continua, (ii) creciente y (iii) cóncava.} Cada agente tendrá una dotación inicial de bienes que podrá consumir e intercambiar. La restricción presupuestaria de cada individuo será determinada por su dotación inicial y el precio de mercado de los bienes que se le haya dotado. Mientras que los precios serán determinados entonces por la demanda (preferencias y poder adquisitivo) y la oferta (dotación total de los bienes). En particular vamos a asumir:

\begin{enumerate}
    \item No hay poder de mercado, los dos agentes en la economía serán tomadores de precios. 
    \item Información completa. Los dos individuos están al tanto de los bienes que hay en la economía, su cantidad y utilidad que les proporciona tanto a el como al otro. 
    \item No hay costos de transacción. 
    \item No hay externalidades. El consumo de un bien no afecta la utilidad de un externo de ninguna manera.\footnote{Este supuesto se puede levantar y llevar al mismo resultado de equilibrio bajo ciertas condiciones, asumiremos que no existe tal posibilidad.} 
\end{enumerate}

Los individuos comerciarán su dotación inicial arbitrariamente asignada llegando a una dotación que sea eficiente. Es intuitivo entender que un individuo solo intercambiará una cantidad de un bien mientras que el bien que le llegue a cambio le genere una utilidad mayor, solo accederan a intercambiar si son más felices con el acuerdo. Por lo tanto, habrá un punto en que dejarán de intercambiarse puesto que se agotaron todas las transacciones posibles, en este equilibrio ninguno de los dos puede estar mejor sin perjudicar al otro.

Este equilibrio se encuentra dentro de lo que se conoce como \textbf{equilibrio walrasiano}.\footnote{También se le llama equilibrio competitivo, fue introducido por Kenneth Arrow and Gérard Debreu a principios de los 50. (Requiere cita)}\marginnote{\textbf{Equilibrio Walrasiano:} Equilibrio de un mercado competitivo, la demanda y oferta de todos los bienes en la economía son iguales.}[-15cm] Cada individuo consumirá con respecto a su restricción, el resultado es que la demanda y la oferta se igualarán, de aquí se suele decir que se \textit{vacía el mercado}. En lo que sigue formalizaremos y pondremos nombre al proceso que se acaba de mencionar.

\subsection{Caja de Edgeworth}

La \textbf{Caja de Edgeworth}\marginnote{\textbf{Caja de Edgeworth:} Herramienta gráfica para representar un modelo de equilibrio general.}[-10cm] será la representación gráfica del modelo. Esta herramienta fue desarrollada por \textbf{Francis Edgeworth}\marginnote{\textbf{Francis Edgeworth:} Economista y estadístico británico que propone las curvas de indiferencia y caja de Edgeworth.}[-7cm] y \textbf{Arthur Bowley},\marginnote{\textbf{Arthur Bowley:} Economista y estadístico británico pionero de usar técnicas de muestreo en encuestas sociales.}[-2cm] economistas y estadísticos británicos. Las dimensiones de esta caja obedecerán a las dotaciones totales de cada bien. 

Esta economía considera dos individuos $i \in A,B$ que se intercambian dos bienes $j \in x,y$. Las dotaciones de bienes que tenga cada individuo se puede describir como un vector $\in \mathbb{R}^2_+$, más precisamente las dotaciones para un individuo $i$ serán:
\begin{equation*}
    \omega_i = (\omega_{ix} , \omega_{iy}) \in \mathbb{R}^2_+
\end{equation*}
En la figura \ref{fig:caja dimensiones} podemos observar que las dimensiones de la caja son $\omega_x = \omega_{Ax} + \omega_{Bx}$ por un lado y $\omega_y = \omega_{Ay} + \omega_{By}$ por el otro. Donde además ubicamos una la dotación inicial como un punto dentro de la caja $\omega^E \in \mathbb{R}^2_+$.
\begin{figure}
    \centering
    \caption{Dimensiones de la caja de Edgeworth}
    \includegraphics[width=\textwidth]{Figuras/EG Dotación inicial.jpeg}
    \label{fig:caja dimensiones}
\end{figure}
Los mercados se encargarán de asignar precios a cada bien dadas las preferencias y dotaciones, estos precios los podemos denotar como un vector $p = (p_x, p_y) \in \mathbb{R}^2_+$. Para el individuo $i$ la riqueza se puede denotar como la cantidad de un bien ponderado por su precio: 
\begin{equation*}
    R_i = p_x \cdot \omega_{ix} + p_y \cdot \omega_{iy}
\end{equation*}
La riqueza vendrá siendo la restricción presupuestaria a la que está sujeta la maximización de utilidad. Es decir, el individuo $i$ maximizará su función $u: \mathbb{R}^2_+ \xrightarrow{} \mathbb{R}_+$ en este caso del tipo Cobb-Douglas consumiendo bienes $x,y$ y su restricción no estará activa mientras consuma menos o igual que su riqueza.
\begin{align*}
    \max_{x,y} &\quad u_i(x,y) = x_i^\alpha y_i^{\beta} \\
    \text{s.a.} &\quad R_i \leq p_x \cdot x_i + p_y \cdot y_i
\end{align*}
Se pueden graficar las curvas de indiferencia dada las preferencias de cada individuo dentro de la caja. Es importante recordar que al hablar de curvas de indiferencia, el moverse a la derecha de la curva de indiferencia es una mejor situación en cuanto a utilidad para el individuo, lo contrario si nos movemos a la izquierda (Desde el punto de vista del individuo $A$, puesto que para el individuo $B$ sería el caso contrario). Podemos caracterizar los puntos en que por lo menos uno de los individuos aumenta su utilidad en el lente de mejorías de pareto, el cual es el área rayada en la figura \ref{fig:caja indiferencias}.
\begin{figure}[h]
    \centering
    \caption{Curvas de indiferencia y lente de mejorías de pareto}
    \includegraphics[width=\textwidth]{Figuras/EG Curvas de indiferencias.jpeg}
    \label{fig:caja indiferencias}
\end{figure}

\subsection{Eficiencia y óptimos de pareto}

Dadas las preferencias y dotaciones los individuos podrán intercambiar bienes para maximizar su utilidad. El intercambio se dará si es que hay una posibilidad de mejora para los dos individuos, cuando todas las oportunidades de mejora se hayan agotado nos encontraremos en un óptimo de pareto. Un \textbf{óptimo de Pareto}\marginnote{\textbf{Óptimo de Pareto:} Punto en que las oportunidades de intercambio están agotadas.}[-4cm] graficamente es un punto en la caja de Edgeworth en que no podemos mejorar a un individuo sin empeorar al otro. Este fue un concepto acuñado por \textbf{Vilfredo Pareto}.\marginnote{\textbf{Vilfredo Pareto:} Economista italiano, adopta los equilibrios walrasianos e incorporó a ellos el concepto de óptimo de Pareto.}[-2cm]

Hay que aclarar que estar en un óptimo no es sinónimo de estar en equilibrio, esto tan solo describe una situación en donde no hay intercambios \textit{win-win}. También se puede comentar que estar en un óptimo de pareto no considera las implicaciones morales en cuanto a equidad. Por ejemplo, una dotación inicial esquina donde $A$ ó $B$ concentra todos los recursos cuenta como un óptimo de Pareto.

La manera de saber si un punto es óptimo de Pareto es si la \textbf{tasa marginal de sustitución}\marginnote{\textbf{Tasa marginal de sustitución:} Unidades de un bien que se están dispuestos a renunciar por una unidad marginal del otro bien, manteniendo el mismo nivel de utilidad.}[-1.5cm] para los dos individuos son iguales (condición \ref{eq:TMS=TMS}). La tasa marginal de sustitución se define como las unidades de un bien que pueden cambiar por una unidad de otro bien manteniendo la utilidad constante.
\begin{equation}
    TMS_{x,y}^A = TMS_{x,y}^B \label{eq:TMS=TMS}
\end{equation}
La tasa marginal se calcula como la derivada parcial de la utilidad respecto a un bien divido por la derivada parcial de la utilidad respecto al otro bien. Por lo que podemos expandir la condición \ref{eq:TMS=TMS} como: 
\begin{align*}
    \frac{    \frac{\partial u_A(x,y)}{\partial x}       }{      \frac{\partial u_A(x,y)}{\partial y}     } & =  \frac{    \frac{\partial u_B(x,y)}{\partial x}       }{      \frac{\partial u_B(x,y)}{\partial y}     } 
\end{align*}
La \textbf{curva de contrato} es la colección de puntos óptimos de pareto dentro de la caja de Edgeworth.\marginnote{\textbf{Curva de contrato:} Colección de puntos óptimos de Pareto dentro de la caja de Edgeworth.} A continuación llegamos a la expresión que denotan todos los óptimos de pareto que componen la curva de contrato para los individuos $A$ y $B$. Siendo que el problema para $A$ se plantea como:
\begin{align*}
    \max_{x,y} &\quad u_A(x,y) = x_A^\alpha y_A^{\beta}\quad \\
    \text{s.a.} &\quad p_x \cdot w_{Ax} + p_y \cdot w_{Ay} = p_x \cdot x_A + p_y \cdot y_A
\end{align*}
El problema para $B$ es simétrico\footnote{Cada vez que se diga que un problema es simétrico significa que la solución algebráica seguirá el mismo proceso, por lo que podemos ocupar en este caso el resultado de $A$ para solucionar directamente el resultado de $B$ sin hacer el proceso algebráico de nuevo.}:
\begin{align*}
    \max_{x,y} &\quad u_B(x,y) = x_B^\alpha y_B^{\beta}\quad \\
    \text{s.a.} &\quad p_x \cdot w_{Bx} + p_y \cdot w_{By} = p_x \cdot x_B + p_y \cdot y_B
\end{align*}
Derivamos para obtener la expresión de la tasa marginal de sustitución para ambos individuos. 
\begin{align*}
    TMS_A  = \frac{\alpha x_A^{\alpha -1} y_A^{\beta}}{\beta x_A^\alpha y_A^{\beta -1} } = \frac{\alpha}{\beta} \cdot \frac{y_A}{x_A} \\
    TMS_B= \frac{\alpha x_B^{\alpha -1} y_B^{\beta}}{\beta x_B^\alpha y_B^{\beta -1} } = \frac{\alpha}{\beta} \cdot \frac{y_B}{x_B} 
\end{align*}
Igualamos,
\begin{align*}
    TMS_A = TMS_B \quad \Longrightarrow \quad& \frac{\alpha}{\beta} \cdot \frac{y_A}{x_A} = \frac{\alpha}{\beta} \cdot \frac{y_B}{x_B} \\
   &  \frac{y_A}{x_A} = \frac{y_B}{x_B}
\end{align*}
Es directo entender que,
\begin{align*}
    x_A + x_B = \omega_x \quad 
    y_A + y_B = \omega_y
\end{align*}
Y por tanto aprovechamos para definir nuestra curva de contrato Reemplazando una de aquellas en la expresión anterior.
\begin{align*}
    \frac{y_A}{x_A} &= \frac{\omega_y - y_A}{\omega_x - x_A} \\
    y_A\omega_x &= x_A\omega_y
\end{align*}
Reordenando la expresión,
\begin{equation}
    x_A \cdot \left( \frac{\omega_y}{ \omega_x} \right) = y_A \label{eq:curva de contrato}
\end{equation}
\begin{figure}[h]
    \centering
    \caption{Curva de contrato}
    \includegraphics[width=\textwidth]{Figuras/EG Curva de contrato.jpeg}
    \label{fig:diapositiva3}
\end{figure}
Es decir, una dotación que cumpla con la expresión \ref{eq:curva de contrato} pertenece a la curva de contrato y por ende es óptimo de pareto, se agotaron las posibilidades de intercambio por lo que no se puede aumentar la utilidad de uno sin reducir la del otro. La curva de contrato va a tener diferentes formas dependiendo de la función de utilidad de los individuos, en algunos casos no habrá.\footnote{Los gráficos estáran en el anexo.} 

\subsection{Demandas Marshallianas}

Los individuos tendrán una \textbf{demanda marshalliana}\marginnote{\textbf{Demandas Marshallianas:} Son las demandas óptimas resultantes de un problema de maximización sujeto a alguna restricción.} acorde a sus preferencias y riqueza. Para esto el individuo maximiza su utilidad sujeto a una restricción. Si resolvemos el problema para A encontraremos la demanda del individuo para cada bien. En este caso diremos que $\alpha,\beta >0$ y que $\alpha + \beta = 1$
\begin{align*}
    \max_{x,y} &\quad u_A(x,y) = x_A^\alpha y_A^{\beta}\quad \\
    \text{s.a.} &\quad R_A = p_x \cdot x_A + p_y \cdot y_A
\end{align*}
Planteando el lagrangeano,
\begin{equation*}
    \mathcal{L}= x_A^\alpha y_A^{\beta} + \lambda (R_A - p_x \cdot x_A - p_y \cdot y_A)
\end{equation*}
Derivamos las condiciones de primer orden.
\begin{align*}
    \frac{\partial \mathcal{L}}{\partial x_A} = \alpha x_A^{\alpha-1}y_A^\beta - \lambda_x p_x = 0 \quad \quad
    \frac{\partial \mathcal{L}}{\partial y_A} = \beta x_A^\alpha y_A^{\beta -1} - \lambda_y p_y = 0
\end{align*}
Se despejan los lambdas y se igualan las expresiones.
\begin{align*}
    & \lambda_x = \frac{\alpha x_A^{\alpha-1}y_A^\beta }{p_x} \quad 
    \lambda_y = \frac{\beta x_A^\alpha y_A^{\beta-1}}{p_y} \longrightarrow \frac{\alpha x_A^{\alpha-1}y_A^\beta }{p_x} = \frac{\beta x_A^\alpha y_A^{\beta-1}}{p_y} \\
    \text{Reescribiendo,}\quad & x_A = y_A  \left( \frac{p_y \alpha}{p_x \beta} \right)  \quad y_A = x_A \left(  \frac{p_x \beta}{p_y\alpha}  \right)
\end{align*}
El siguiente paso es reemplazar alguna de las expresiones anteriores en la restricción:
\begin{align*}
    p_x\left( y_A\frac{p_y\alpha}{p_x\beta} \right) + p_yy_A = R_A
\end{align*}
Por lo tanto las demandas marshallianas de $A$ serán expresadas como:
\begin{align*}
    y_A^* = \frac{R_A}{p_y} \beta & \quad  x_A^* = \frac{R_A}{p_x} \alpha
\end{align*}
Como el problema es simétrico para $B$ podemos directamente decir:
\begin{align*}
    y_B^* = \frac{R_B}{p_y} \beta & \quad  x_B^* = \frac{R_B}{p_x} \alpha
\end{align*}
Las demandas marshallianas estarán en función de la riqueza, el precio del bien y las preferencias.

\subsection{Precios relativos}

Los precios en esta economía están determinados por la dotación y preferencias y serán tales que todos los mercados estén en equilibrio. Lo que importa no son los los valores absolutos que se les da a los precios sino el el precio de un bien en relación al otro, es por esto que un mismo equilibrio se puede sostener con un vector precios $p=(1,2)$ como con un vector $p=(2,4)$ (si bien los valores absolutos cambiaron, la relación 1 es a 2 sigue presente). Es por esto que al resolver el modelo podemos normalizar uno de los precios a 1, de esta manera  simplificamos el proceso.\footnote{El tener precios numerarios es lo mismo que tomar el vector precios $p = (p_i,p_j)$ y multiplicarlos por $1/p_i$, dejándolos como $p = (1, \frac{p_j}{p_i})$. La gracia está en que tanto la primera expresión del vector como la segunda funcionan para un mismo equilibrio.} 

Para obtener los precios podemos apelar a la ley de Walras y proponer la siguiente expresión,
\begin{equation*}
    x_A^* + x_B^* = \omega_x, \quad \quad y_A^* + y_B^* = \omega_y
\end{equation*}
Tomando el caso del bien $x$ reemplazamos las demandas marshallianas.
\begin{align*}
   \frac{R_A}{p_x} \alpha + \frac{R_B}{p_x} \alpha  &= \omega_x \\
   \frac{\alpha}{p_x} (R_A + R_B) &= \omega_x
\end{align*}
Despejando $p_x$ de esta expresión y (por simetría) expresando también $p_y$.
\begin{align*}
    p_x = (R_A + R_B)\frac{\alpha}{\omega_x}, \quad \quad p_y = (R_A + R_B)\frac{\beta}{\omega_y}
\end{align*}
Lo cual hace sentido, a mayor preferencia por determinado bien mayor es el precio, por otro lado a menor dotación del bien mayor su precio. Por convención diremos que la riqueza total se puede escribir como $R_A + R_B = R = p_x\omega_x + p_y\omega_y$. Dada esta expresión tendremos un problema, todavía no hemos despejado los precios puesto que estos dependen de la riqueza la cual depende recursivamente de los precios. Podemos hacer dos cosas, una de ellas es normalizar uno de los precios a 1, y la otra opción es dividir directamente $p_y/p_x$ para obtener $p_y$ en relación a $p_x$. 

Los precios relativos se pueden expresar entonces como:
\begin{align*}
    \frac{p_y}{p_x} = \frac{ \frac{R\alpha}{\omega_x}  }{  \frac{R\beta}{\omega_y}  } = \frac{\beta \omega_x}{\alpha \omega_y}
\end{align*}
Son las mismas leyes de oferta y demanda, a mayor preferencia de $y$ en relación a $x$ mayor será el precio de $y$ relativo al precio de $x$. 

\subsection{Teoremas del bienestar}

\textbf{Primer teorema del bienestar:} Todo equilibrio walrasiano es óptimo de pareto.

Este teorema plantea que si hay un equilibrio en la economía no habrán oportunidades de intercambio.\footnote{Demostrado en Arrow, Kenneth J. and Gerard Debreu, “Existence of Equilibrium for a Competitive Economy,” Econometrica, 1954.} Este teorema se relaciona fuertemente con la mano invisible en donde cada individuo buscando su propio beneficio llegaría a un punto eficiente. Recordar que para llegar a este equilibrio se tienen que cumplir supuestos bastante fuertes. 

\textbf{Segundo teorema del bienestar:} Todo óptimo de Pareto puede conseguirse mediante un equilibrio walrasiano. 

Para cualquier óptimo de Pareto podemos redistribuir las dotaciones iniciales de manera de conseguir un equilibrio walrasiano. Dada una nueva distribución inicial podemos llegar a un óptimo de Pareto dado mediante el equilibrio walrasiano. 

\subsection{Equilibrio con producción}

Una extensión de lo visto hasta ahora sería considerar que los bienes consumidos se tienen que producir. En este caso las dotaciones no van a ser de bienes sino de factores de producción capital y trabajo ($K,L$). Habrá un total de capital de $\bar{K}$ y un total de trabajo $\bar{L}$, con los que se producen los bienes $x,y$. 

Análogo al consumo la condición de eficiencia para la producción sigue siendo la igualdad entre las tasas marginales, en este caso la \textbf{tasa marginal técnica} (TMT) de sustitución. Es decir, la producción de un bien que podemos intercambiar por la producción adicional de una unidad del otro,
\begin{equation}
    TMT = \frac{ \frac{\partial f(K,L)}{\partial K} }{\frac{\partial f(K,L)}{\partial L} } \label{eq: TMT dfn}
\end{equation}
Dado que se requieren capital y trabajo para los dos bienes estos se tendrán que repartir las dotaciones de manera eficiente, siguiendo \ref{eq: TMT = TMT}.
\begin{equation}
    TMT^x_{K,L} = TMT^y_{K,L} \label{eq: TMT = TMT}
\end{equation}
Podemos formar una caja de Edgeworth de producción como se puede ver en la figura \ref{fig:diapositiva4}.
\begin{figure}[h]
    \centering
    \caption{Caja de Edgeworth en la producción}
    \includegraphics[width=\textwidth]{Figuras/EG Curva de contrato con producción.jpeg}
    \label{fig:diapositiva4}
\end{figure}
Las curvas de contrato se relacionan con las fronteras de posibilidades. La \textbf{frontera de posibilidades}\marginnote{\textbf{Frontera de posibilidades:}} es una curva que describe todos los puntos en donde se ocupan eficientemente los factores productivos para producir dos bienes, es decir, son los puntos factibles donde no es posible aumentar la producción de un bien sin disminuir la producción del otro. En la figura \ref{fig:diapositiva5} se observan dos puntos, uno que está ubicado en la curva es eficiente y otro que no es eficiente puesto que se puede producir más de un bien sin disminuir la producción de otro. Como todos los puntos de la curva de contrato satisfacen la condición de eficiencia se puede decir que también pertenecen a la curva de contrato.
\begin{figure}[h]
    \centering
    \caption{Frontera de posibilidades de producción}
    \includegraphics[width=\textwidth]{Figuras/EQ Frontera de posibilidades de producción.jpeg}
    \label{fig:diapositiva5}
\end{figure}
Para más detalle puede leer el anexo.

\section{Poder de mercado}

\section{Bienes club, comunes y públicos}

\section{Elección pública}

\section{Externalidades}

/section{Externalidades}

\subsection{Motivación}
Al abordar la temática de externalidades comenzamos a salir del marco walrasiano, si bien está es conocida como una falla de mercado clásica, es muy interesante estudiarla desde la economía política ya que existen distintas soluciones que se relacionan con tendencias ideologicas.

Consideramos como externalidad\footnote{Para efectos de este curso nos centraremos en la externalidades de equilibrio parcial.} la situación donde, ya sea la producción o consumo de un bien, tiene efectos en el bienestar de un tercero que no participa de la interacción de mercado. Entre los ejemplos típicos encontramos las externalidades ambientales, el caso del fumador, aumento de la educación, entre otros. Esta falla de mercado nos saca del marco walrasiano ya que desde el punto de vista social no tenemos un mercado en equilibrio.

\subsection{Externalidad en la producción}

Para analizar el funcionamiento de las externalidades negativas en la producción, utilizaremos como benchmark el equilibrio parcial de un mercado en competencia perfecta. Este modelo de mercado se compone por una oferta y demanda privada, las cuales en su interacción nos entregan un precio y una cantidad de equilibrio.

La oferta privada en este mercado está definida por el costo marginal de producción del bien ($Cmg$) percibidos por quien produce, estos pueden ser costos provenientes de la compra de insumos, gastos en energía, arriendo de un local, etc. 

\begin{figure}[htbp]
    \centering
    \caption{Equilibrio parcial en competencia perfecta}
    \includegraphics[width=0.5\textwidth]{Figuras/Eq Parcial EXT.jpg}
    \label{fig:Eq parcial}
\end{figure}

Para añadir externalidades negativas a este modelo, debemos considerar que existen costos adicionales que no son percibidos ni reconocidos por quién produce, por lo tanto tampoco son reflejados en la curva de oferta privada, estos nuevos costos los llamaremos ``costos sociales''. Con estos mayores costos, obtenemos que en equilibrio deberíamos observar un precio mayor y una cantidad producida menor.

\begin{figure}[htbp]
    \centering
    \caption{Externalidad en la producción}
    \includegraphics[width=0.5\textwidth]{Figuras/Externalidad Produccion.jpg}
    \label{fig:Ext. Producción}
\end{figure}

\subsubsection{Externalidad en la demanda}

Para el caso de externalidades positivas el problema no radica en costos no considerados, sino que existe una demanda social que es mayor a la privada. Para analizar este tipo de externalidad nuevamente utilizaremos como benchmark el equilibrio parcial de un mercado en competencia perfecta.

Con lo anterior tenemos que la demanda privada representa la disposición a pagar por el consumo individual de un bien, esto genera que se subestimen los beneficios que tiene sobre la sociedad el consumo de dicho bien. Por otro lado, la demanda social si refleja de manera fidedigna los beneficios generados a nivel social por el consumo del bien, la internalización de estos beneficios resultaría en una mayor disposición a pagar.

El resultado del equilibrio en el mercado considerando la demanda social nos entrega un mayor precio y cantidad producida.

Las externalidades negativas tambien se pueden representar por el lado de la demanda, esto ocurre cuando el efecto en el bienestar del tercero viene dado por el consumo del bien más que por su producción, ejemplo de esto es el consumo del cigarro, con esto tenemos que quien consume no internaliza la desutilidad que genera en los demás agentes y por lo tanto no representa la disposición a pagar de la sociedad. El equilibrio del mercado resulta diferente para este caso, ya que tenemos que en equilibrio el precio y la cantidad producida disminuyen

\subsubsection{Solución Pública}

La solución de carácter pública para corregir esta falla de mercado fue creada por el economista \textbf{Arthur Pigou}\marginnote{\textbf{Arthur Pigou:}(1877-1959) Economista inglés de la Universidad de Cambridge, conocido por inventar el término \textit{externalidades} y contribuir con su solución pública.}, los mecanismos utilizados por esta solución los conocemos como \textit{impuestos y subsidios pigouvianos}, y los supuestos utilizados son que el estado tiene información completa acerca de las preferencias de la sociedad y además la capacidad suficiente para que sus impuestos o subsidios modifiquen el comportamiento de los agentes.

Es importante conocer la diferencia entre tipo de impuestos, de manera general en el estado encontramos impuestos utilizados con el fin de recaudar y subsidios que se enfocan en redistribuir, por otra parte tenemos los impuestos y subsidios pigouvianos que corrigen externalidades mediante un cambio en el comportamiento de los agentes. Dentro de los ejemplos clásicos de impuestos y subsidios pigouvianos encontramos el impuesto al tabaco y los colegios subvencionados.

\subsubsection{Medida de efectividad}

Para poder medir la efectividad de la solución pública utilizaremos como herramienta la \textit{evaluación social de proyectos públicos}, esta es un área de la economía desarrollada por \textbf{Arnold Harberger}\marginnote{\textbf{Arnold Harberger:}(1924-) Economista estadounidense de la Universidad de Chicago, conocido por ser pionero en el estudio de la evaluación social de proyectos públicos}. Considerando que la aplicación de impuestos o subsidios es costosa (Ej: costos de administración), la evaluación social consiste en comparar las preferencias sociales y privadas, para determinar la magnitud de la \textit{pérdida social}\footnote{En inglés se conoce como \textit{dead weight loss (DWL)}} existente en un mercado, representada gráficamente por los \textit{triángulos de Harberger}, con esto podremos comparar los costos y beneficios de resolver una externalidad y usar esto como medida de eficiencia. 

\section{Fallas del Estado}

\section{Trampas de ineficiencia y trampas de ineficacia}

\begin{appendices}
\input{Tex/Apéndices/Ap El problema de lo privado y lo público}
\end{appendices}


\chapter{El problema del tiempo y del riesgo}

\input{Tex/Optimización intertemporal y la aversión al cambio}

\section{Inconsistencia temporal}

\section{Optimización interestados y la aversión al riesgo}

\section{Seguros}

\input{Tex/La teoría de mercados eficientes}

\section{La teoría de la protección social}

\section{Trampas de impaciencia y trampas de miedo}

\begin{appendices}
\chapter{Apéndice}

\section{Random Walk}

Incluyendo incertidumbre a un modelo con expectativas racionales podemos obtener distintos resultados dependiendo del tipo de función de utilidad. En este caso veremos como un individuo neutro al riesgo sigue una trayectoria de consumo tipo \textit{random walk}.

Primero resolvamos el problema con un consumo futuro incierto.
\begin{align*}
    \max _{c_t , c_{t+1}} \quad  u(c_t)+ \mathbb{E}_t \{u(c_{t+1})\} \\ 
    s.a \quad y_t + \frac{y_{t+1}}{1+r} = c_t +\frac{c_{t+1}}{1+r}
\end{align*}

Planteamos el lagrangeano:
\begin{equation*}
    \mathcal{L}: u(c_t)+ \mathbb{E}_t \{ u(c_{t+1}) \} + \lambda \left(y_t + \frac{y_{t+1}}{1+r} - c_t - \frac{c_{t+1}}{1+r} \right)
\end{equation*}

Derivamos las condiciones de primer orden para encontrar la ecuación de Euler del consumo. 
\begin{align}
    \frac{\partial \mathcal{L}}{\partial c_t} =u'(c_t) -\lambda = 0    \notag \\   u'(c_t) = \lambda \\
    \frac{\partial \mathcal{L} }{\partial c_{t+1}} = \frac{1}{1+\rho} \mathbb{E}_t \{ u'(c_{t+1})\} - \frac{\lambda }{1+r} = 0 \notag 
 \\ \frac{1+r}{1+\rho } \mathbb{E}_t \{u'(c_{t+1})\} = \lambda \\
 u'(c_t) = \frac{1+r}{1+\rho} \mathbb{E}_t\{u'(c_{t+1})\} \label{eq: Ecuación de euler random}
\end{align}

La expresión \ref{eq: Ecuación de euler random} es la ecuación de Euler.

Para probar que un individuo neutral al riesgo sigue un \textit{random walk} en este caso podemos reemplazar la función de utilidad con una función cuadrática como en la expresión \ref{eq: ecuación cuadrática}. La utilidad marginal por tanto va a ser lineal lo cual describe un individuo neutro al riesgo.
\begin{equation}
    u(c_t) = bc_t - \frac{a}{2}c_t ^2 \label{eq: ecuación cuadrática}
\end{equation}

Remplazmaos las funciones de utilidad marginales en la ecuación de Euler del consumo y asumimos que $r=\rho = 0$:
\begin{align*}
    b-ac_t = \mathbb{E}_t\{b-ac_{t+1}\}
\end{align*}

Aplicamos propiedades de la esperanza y simplificamos los $b$ y $-a$:
\begin{equation}
    c_t = \mathbb{E}_t \{c_{t+1} \} \label{eq: random walk resultado}
\end{equation}
Según muestra la ecuación \ref{eq: random walk resultado} el consumo en $t$ va a ser igual a lo que se espera que sea el consumo en $t+1$. Por lo que la trayectoria de consumo se mantendrá igual hasta que llegue un shock que la desvíe.

\section{Incertidumbre y ahorro por precaución}

Bajo el mismo caso de incertidumbre, un individuo con aversión al riesgo tendrá un ahorro por precaución. Esto ya que no se cumple el caso de \ref{eq: random walk resultado} ya que tendremos utilidades marginales crecientes a tasas decrecientes. 

Esto se traduce en que la esperanza de la utilidad es mayor a utilidad de la esperanza
\begin{equation}
    \mathbb{E}_t \{ u(c_t) \} < u(\mathbb{E}_t \{ c_t \})
\end{equation}
Podemos derivar la expresión. 
\begin{equation}
    \mathbb{E}_t \{ u'(c_t) \} > u'(\mathbb{E}_t \{ c_t \})
\end{equation}
Según \ref{eq: Ecuación de euler random} (con $\rho = r = 0$) podemos reemplazar.
\begin{align}
    u'(c_t) = u'(\mathbb{E}_t \{ c_{t+1} \}) < \mathbb{E}_t\{ u'(c_{t+1}) \} \notag \\
    u'(c_t) < \mathbb{E}_t\{ u'(c_{t+1}) \} \label{eq: incertidumbre marginla averso}
\end{align}
De \ref{eq: incertidumbre marginla averso} podemos interpretar que debido al perfil del consumidor se consume menos por una necesidad de ahorro por precaución.

\section{Modelo CAPM de consumo}
Los retornos están sujetos a varianza. retornos puedes estar sujetos a varianza (incertidumbre). Para individuos con expectativas racionales que tienen la opción de consumir en el presente o de invertir para llevar dinero al futuro podremos llegar a la ecuación de euler del tipo \ref{eq: Ecuación de euler random}. Específicamente,
\begin{equation}
    u'(c_1) = \mathbb{E}_1  \left[  \frac{1+r^i}{1+\rho} u'(c_2)  \right] \label{eq: Ecuación de euler random 2}
\end{equation}
Donde $r^i$ es un activo con riesgo. Queremos buscar el diferencial entre la esperanza del activo riesgoso y el libre de riesgo. Utilizando las propiedades de la esperanza podemos desarrollar sacando el descuento intertemporal de la esperanza y multiplicando la esperanza de los retornos $1+r^i$ y $u'(c_2)$. Recordamos como desarrollar $\mathbb{E}(X\cdot Y)$ y aplicamos a \ref{eq: Ecuación de euler random 2}
\begin{align}
    \mathbb{E}(X\cdot Y) = \mathbb{E}(X) \cdot \mathbb{E}(Y) + \text{Cov}(XY) \\
    u'(c_1) = \frac{1}{1+\rho} \left[ \mathbb{E}_1(1+r^i) \cdot \mathbb{E}_1 (u'(c^2)) + \text{Cov}(1 + r^i, u'(c_2)       \right] \\
    \mathbb{E}_1 (1+r^i) = \frac {1}{\mathbb{E}_1 (u'(c_2))} \left[   (1+\rho)u'(c_1) - \text{Cov}(1+r^i, u'(c_2))     \right] \label{eq: Diferencial riesgoso}
\end{align}

Ahora tenemos la esperanza de los activos riesgosos, necesitamos encontrar el diferencial entre estos y los de libre riesgo. Sabemos que en caso de que $1+r$ sea libre de riesgo podremos sacar los retornos como constantes de  \ref{eq: Ecuación de euler random 2}. Llegamos a lo siguiente,
\begin{equation}
    1+r = \frac{1}{\mathbb{E}_1 (u'(c_2))} [(1+\rho)u'(c_1)] \label{eq: diferencial libre de riesgo}
\end{equation}
Por lo que ahora que tenemos las variables definidas podemos encontrar el diferencial restando \ref{eq: Diferencial riesgoso} con \ref{eq: diferencial libre de riesgo}. 
\begin{equation*}
    \mathbb{E}_1 (1+r^i) - (1+r) = \frac {(1+\rho)u'(c_1) - \text{Cov}(1+r^i, u'(c_2)) - (1+\rho)u'(c_1) }{\mathbb{E}_1 (u'(c_2))}  
\end{equation*}
Llegamos a que la diferencia entre los retornos riesgosos y los libres de riesgos debieran seguir esta igualdad.
\begin{equation}
    \mathbb{E}_1(r^i) - r = - \frac{\text{Cov}(1+r^i, u'(c_2))}{\mathbb{E}_1(u'(c_2))}
\end{equation}
Esta expresión es la que da sentido a los premios por riesgo de un activo con retornos inciertos. Aunque se parezca mucho al consumo intertemporal este modelo CAPM de consumo considera incertidumbre. Por lo que las personas no solo estarían decidiendo si invertir a futuro por precaución sino también por aumentar el consumo futuro bajo un riesgo. 

En caso de que si $\text{Cov}(1+r^i,u'(c_2))<0$ habrá un premio de por riesgo. Recordando que $u'(c_2)<0$ podemos inferir que el retorno covaría positivamente con el consumo, por lo que en el futuro $t=2$ se necesitaría pagar un premio por riesgo. Esto ya que si un activo paga más cuando el consumo es alto no estaría proveyendo de un seguro contra las caídas del ingreso, la única razón de mantenerlo en el portafolio sería si provee un buen retorno. 

En caso de que $\text{Cov}(1+r^i,u'(c_2))>0$, es decir el retorno es bajo cuando el consumo es alto el rendimiento será menor al libre de riesgo. Esto es porque además de servir como ahorro también juega una función de seguro contra los malos tiempos

% desarrollar mejor, esto está en los ppt de macroeconomía I y falta mucho detalle...

% Introducir de mejor manera el capm
\end{appendices}

% Creo provisoriamente este capítulo para incluir apuntes de organización industrial que pueden ser utiles para otras secciones
\chapter{Organización industrial}

\subsection*{Introducción a la teoría de juegos}
La teoría de juegos es una manera de modelar distintos tipos de interacciones de competición, cooperación y otras. Modelar qué decisión y el por qué puede ser tan simple como complejo dependiendo de los factores que que consideremos relevantes. En este capítulo se trataran los juegos simultáneos en secuencia e iterados (también llamados repetidos). 

\section{Juegos simultaneos}

En juegos simultáneos los agentes toman decisiones al mismo tiempo, estas decisiones interactuan resultando en \textbf{pagos}. El pago a cada persona refleja un nivel de utilidad, por lo que cada agente buscará llegar a un resultado del juego (la combinación de acciones) que maximize su pago. 

Para dos agentes $A$ y $B$ cada uno toma la decisión $X$ o $Y$, la combinación de decisiones llevará a cierto nivel de pagos. Este escenario puede ser representado por una \textbf{matriz de pago}\marginnote{\textbf{Matriz de pago:} La matriz de pago es una representación de los pagos un juego simultáneo para un número definido de individuos y estrategias.}[-3cm] en el cuadro \ref{cuadro: matriz de pago genérica}.

\begin{table}[!htbp]
  
  \centering
  \caption{Matriz de pagos}
  \setlength{\extrarowheight}{2pt}
  \label{cuadro: matriz de pago genérica}
  \begin{tabular}{*{4}{c|}}
    \multicolumn{2}{c}{} & \multicolumn{2}{c}{$B$}\\\cline{3-4}
    \multicolumn{1}{c}{} &  & X  & Y \\\cline{2-4}
    \multirow{2}*{$A$}  & X & $(a,b)$ & $(c,d)$ \\\cline{2-4}
    & Y & $(e,f)$ & $(g,h)$ \\\cline{2-4} 
  \end{tabular}
\end{table}

Ejemplifiquemos cómo se lee la tabla. Si $A$ tomó la estrategia $X$ y $B$ toma la decisión $Y$ entonces los pagos correspondientes son $(c,d)$, donde $c$ es el pago para $A$ y $d$ el pago para $B$. De la misma manera si $A$ elije $Y$ y $B$ decide $Y$ la matriz de pagos será $(g,h)$. 

Hay un componente estratégico en la interacción de $A$ y $B$ pues los pagos que reciba cada uno dependerá de la decisión que tome el otro. Por ejemplo si $A$ sabe que $B$ decide $X$ entonces $A$ elegirá la estrategia que maximize sus pagos. Específicamente $A$ está decidiendo con respecto a los pagos en negrita en el cuadro \ref{cuadro: B decide X}.

\begin{table}[!htbp]
  \centering
  \caption{Matriz de pagos con $B$ decidiendo $X$} \label{cuadro: B decide X}
  \setlength{\extrarowheight}{2pt}
  \begin{tabular}{*{4}{c|}}
    \multicolumn{2}{c}{} & \multicolumn{2}{c}{$B$}\\\cline{3-4}
    \multicolumn{1}{c}{} &  & X  & Y \\\cline{2-4}
    \multirow{2}*{$A$}  & X & $(\textbf{a,b})$ & $(c,d)$ \\\cline{2-4}
    & Y & $(\textbf{e,f})$ & $(g,h)$ \\\cline{2-4}  
  \end{tabular}
\end{table}

La decisión que tome $A$ dependerá qué pago es mayor, $a$ ó $e$, si es $a$ entonces elije $X$ y caso contrario elije $Y$.

\subsection{Estrategias y resolución de juegos}

\subsection{Equilibrios de Nash y óptimos de pareto en juegos}

\subsection{Juegos canónicos}

\subsubsection*{Dilema del prisionero}

El dilema del prisionero describe la situación en que dos criminales sospechosos son detenidos y separados para un proceso de interrogación. Si uno de los dos delata al otro y su complice no, este último tendrá pena de cárcel y el delator quedará libre. En caso de que los dos se delaten entre sí, ambos son condenados a años de cárcel. Por útlimo si ninguna se delata entre sí, quedan libres o cumplen penas menores.

La matriz de pago específicamente se en el cuadro. A mayor pena menor es el pago. 

\begin{table}[!htbp]
    \centering
    \caption{El dilema del prisionero}
    \setlength{\extrarowheight}{2pt}
    \begin{tabular}{*{4}{c|}}
      \multicolumn{2}{c}{} & \multicolumn{2}{c}{Prisionero $B$}\\\cline{3-4}
      \multicolumn{1}{c}{} &  & Cooperar  & Delatar \\\cline{2-4}
      \multirow{2}*{Prisionero $A$}  & Cooperar & $(-2,-2)$ & $(-10,-1)$ \\\cline{2-4}
      & Delatar & $(-1,-10)$ & $(-6,-6)$ \\\cline{2-4}
    \end{tabular}
  \end{table}

Cada uno tiene una estrategia dominante en delatar al otro, el resultado es un equilibrio de Nash sub-óptimo en términos de Pareto. 

El juego describe como dos personas no cooperan incluso si ello va en contra del interés de las dos. 

\subsubsection*{Mano invisible}

Si se deja la autorregulación de los mercados, dado que los individuos persiguen su propio interés, se produce un equilibrio que es un óptimo de Pareto. Una matriz que lo representa puede ser.

El agente $A$ tendrá una estrategía dominante, la cual será la opuesto a la del otro agente $B$. Finalmente el equilibrio de Nash es eficiente paretianamente.

\begin{table}[!htbp]
    \centering
    \caption{La mano invisible}
    \setlength{\extrarowheight}{2pt}
    \begin{tabular}{*{4}{c|}}
      \multicolumn{2}{c}{} & \multicolumn{2}{c}{Agente $B$}\\\cline{3-4}
      \multicolumn{1}{c}{} &  & $X$  & $Y$ \\\cline{2-4}
      \multirow{2}*{Agente $A$}  & $X$ & $(0,10)$ & $(1,1)$ \\\cline{2-4}
      & $Y$ & $(11,11)$ & $(10,0)$ \\\cline{2-4}
    \end{tabular}
  \end{table}

(Cita Adam Smith)

\subsubsection*{Guerra de los sexos}

Una pareja heterosexual se encuentra incomunicada en medio de un festival de metal. El hombre es fan de Metallica y la mujer fan de Megadeth. 

\begin{table}[!htbp]
    \centering
    \caption{Guerra de los sexos}
    \setlength{\extrarowheight}{2pt}
    \begin{tabular}{*{4}{c|}}
      \multicolumn{2}{c}{} & \multicolumn{2}{c}{Mujer}\\\cline{3-4}
      \multicolumn{1}{c}{} &  & Metallica  & Megadeth \\\cline{2-4}
      \multirow{2}*{Hombre}  & Metallica & $(2,1)$ & $(1,1)$ \\\cline{2-4}
      & Megadeth & $(0,0)$ & $(1,2)$ \\\cline{2-4}
    \end{tabular}
  \end{table}

  Cada uno tiene estragia dominante por ir al concierto de su banda preferida idependiente de lo que haga el otro, hay dos equilibrios de Nash por lo tanto este juego tiene ser resuelto con estrategias mixtas.

\subsubsection*{La caza del venado}
Dos individuos van a cazar ya sea conejos o venados y deben escoger su presa sin conocer la elección del otro cazador. Para cazar el venado (un premio mayor) requieren de la ayuda del otro, mientras que un conejo puede ser cazado por cada uno. Por lo tanto, si cooperan cazando al venado podrán ambos obtener más beneficios y estár en un óptimo de Pareto.

\begin{table}[!htbp]
    \centering
    \caption{La caza del venado}
    \setlength{\extrarowheight}{2pt}
    \begin{tabular}{*{4}{c|}}
      \multicolumn{2}{c}{} & \multicolumn{2}{c}{Cazador $B$}\\\cline{3-4}
      \multicolumn{1}{c}{} &  & Venado  & Conejo \\\cline{2-4}
      \multirow{2}*{Cazador $A$}  & Venado & $(4,4)$ & $(0,3)$ \\\cline{2-4}
      & Conejo & $(3,0)$ & $(3,3)$ \\\cline{2-4}
    \end{tabular}
  \end{table}
En este caso no existen estrategias dominantes, lo que lleva a la existencia de dos equilibrios de Nash, cazar al venado juntos domina paretianamente a cazar conejos juntos. Lo anterior representa un problema de cooperación social y una dicotomía entre seguridad y cooperación. 
\subsubsection*{Chicken}

Se trata de un juego para determinar quien es el más valiente, dos personas se posicionan con sus autos en dos extremos y aceleran de manera que llegará un punto en que choquen entre si. El primero que doble para evitar el impacto es un gallina, dejando al ganador como valiente. Hay tres escenarios, el primero en que uno de los dos dobla y queda como gallina, un segundo escenario en donde los dos doblan ambos quedando como gallinas y por último el caso en que chocan. 

Ambos corredores quieren hacer lo opuesto a lo que haga el otro, los equilibrios de Nash serían entonces los que uno de ellos dobla y el otro sigue. Lo cual puede expresarse por medio de la matriz de pago.\footnote{Rising, L: The Patterns Handbook: Techniques, Strategies, and Applications, page 169. Cambridge University Press, 1998. \textbf{Schedule Chicken}.}
\begin{table}[!htbp]
    \centering 
    \caption{Chicken}
    \setlength{\extrarowheight}{2pt}
    \begin{tabular}{*{4}{c|}}
      \multicolumn{2}{c}{} & \multicolumn{2}{c}{Corredor $B$}\\\cline{3-4}
      \multicolumn{1}{c}{} &  & Ceder (gallina)  & Seguir (valiente) \\\cline{2-4}
      \multirow{2}*{Corredor $A$}  & Ceder (gallina) & $(2,2)$ & $(1,3)$ \\\cline{2-4}
      & Seguir (valiente) & $(3,1)$ & $(0,0)$ \\\cline{2-4}
    \end{tabular}
  \end{table}

\section{Juegos secuenciales}

Los juegos secuenciales son una forma extendida de lo que se ha aplicado hasta ahora. Antes las estrategias eran acciones individuales; confesar, delatar, cooperar, tracionar, etcéra, ahora serán planes completos de acciones. 

Lo ejemplos más usados son respecto al uso de bombas nucleares durante la guerra fría, también aprovechemos recordar como se planteaban estos juegos. Nuestro juego parte de una situación en que la Unión Soviética exige a las potencias
occidentales que abandonen Berlín. En este punto, Estados Unidos tiene dos posibilidades:
Aceptar y No aceptar.

\begin{center}
    \includegraphics[scale=0.5]{Figuras/juegos secuencia.png}  
\end{center}
Las estrategias serán un plan de acción, en esta caso serían (Acepta, No Acepta y Guerra, No Acepta y Bloqueo).

Para resolver estos juegos uno tiene que seguir un método inductivo, resolver desde el futuro hacia el pasado. ¿Por qué resolver de adelante hacia atrás? Porque si resolvieramos como en juegos simultáneos no estaríamos contando si la amenaza (en este caso una respuesta bélica) es creíble. 

Si plantearamos de manera simultánea encontramos dos EN, uno (el de Acepta y Guerra) en que \textbf{no es secuencialmente racional}, para que ese equilibrio sea posible Estados Unidos tiene que aceptar, sin embargo veremos a continuación por qué si Estados Unidos no acepta la Unión Soviética nunca responderá con una respuesta bélica.

\begin{table}[!htbp]
    \centering
    \caption{Dominio sobre Berlín}
    \setlength{\extrarowheight}{2pt}
    \begin{tabular}{*{4}{c|}}
      \multicolumn{2}{c}{} & \multicolumn{2}{c}{Unión Soviética}\\\cline{3-4}
      \multicolumn{1}{c}{} &  & Guerra & Bloqueo \\\cline{2-4}
      \multirow{2}*{Estados Unidos}  & Acepta & $(\underbar{-5},\underbar{10})$ & $(-5,10)$ \\\cline{2-4}
      & No acepta & $(-10,-10)$ & $(\underbar{0},\underbar{0})$ \\\cline{2-4}
    \end{tabular}
  \end{table}

El juego tiene que ser \textbf{secuencialmente racional} por lo que resolvemos de adelante hacia atrás. El resultado será un equilibrio perfecto en subjuegos (EPS).

Por lo tanto primero se resuelve la decisión de la Unión Soviética, la cual debiese decidir bloquear, y luego evaluar que decidirá Estados Unidos, la cual debiese no aceptar. 

\begin{center}
    \includegraphics[scale = 0.50]{Figuras/Inducción.png}
\end{center}

\subsubsection*{Relación EN y EPS}

En un juegos secuencial puedes obtener EN que no sean coherentes con las amenazas creíbles. Los EPS siempre son coherentes con la credibilidad de las amenazas. \textbf{Un EN no siempre es un EPS, pero un EPS siempre es un EN}. 

\section{Competencia imperfecta}

Los \textbf{oligopolios},\marginnote{\textbf{Oligopolio:} Mercado con un número reducido de firmas las cuales pueden incidir en el precio.}[0cm] también llamados mercados de competencia imperfecta, son un escenario en el cual un número reducido de empresas inciden en cierta medida en el precio de mercado. Es por este control parcial sobre el precio que se le podría considerar un entremedio entre el poder de mercado de un monopolio y de una firma en competencia perfecta.

Debido a que todas las firmas afectan el precio habrá \textbf{interdependencia monopolística},\marginnote{\textbf{Intedependencia monopolística:} De manera más general llamado interdependencia estratégica. Las empresas toman sus decisiones formando creencias de lo que hará su competencia.}[-1cm] lo cual sugiere que hay un factor estratégico en las competencias olipólicas. Por ejemplo, si una firma fija cierto precio, su rival puede responder con un precio menor para quedarse con una mayor demanda.\footnote{Esto asumiendo que son bienes perfectamente sustitutos (producto homogéneo).} También, si se cree que la firma rival va a producir mucho del bien, a las demás firmas les conviene producir menos para no desplomar el precio de mercado por un exceso de oferta. Es decir, las decisiones de una firma afectan a su competencia y viceversa, ante esto las firmas formarán creencias de lo que hará la competencia para tomar sus propias decisiones. 

\subsubsection*{La estrategia desde la teoría de juegos}

La manera en que entendemos estas interacciones propias de la \textbf{organización industrial}\marginnote{\textbf{Organización industrial:} Área de la teoría de la firma que se enfoca en la estructura e interacciones en los mercados.} es mediante la teoría de juegos. Tal como se mencionaba antes, en la teoría de juegos los jugadores forman creencias de las estrategias del otro con tal de reaccionar de la mejor manera. A continuación plantearemos como se verían estos juegos aplicado a los distintos tipos de competencia imperfecta: por cantidades y por precios. 

Para esta ocasión solo se abarcarán juegos normales (simultáneos), es decir, los jugadores $i \in 1,2,\ldots,N$ serán racionales que deciden su acción o combinación de acciones $a_i \in A_i$ resultando en un pago $\pi_i(a)$ para cada firma.

Las firmas elegiran $a_i$ de manera de maximizar sus pagos, una estrategia será mejor que otra mientras el pago resultante sea mayor. Dado que los rivales $-i$ escogen una estrategia $a_{-i}$, la firma $i$ tendrá una respuesta óptima $a_i^*$ en que los pagos sean mayores, es decir,\footnote{Dada las características del juego y considerando que los individuos son racionales es esperable que siempre elijan la respuesta óptima $a_i^*$.} 
\begin{equation}
    \pi_i(a_i^*, a_{-i}) \geq \pi_i(a_i, a_{-i}), \forall a_i . \label{eq: mejor respuesta}
\end{equation}
Esto es equivalente a decir que jugando piedra papel o tijera, si mi rival elige tijera la mejor respuesta para el papel es la tijera, mientras que la mejor respuesta para la tijera es la piedra. Podremos denotar la mejor respuesta de $i$ en función de la estrategia de los demás como: $a^*_i(a_{-i})$. Esta \textbf{función de reacción}\marginnote{\textbf{Función de reacción:} Función que describe la mejor forma de responder ante las decisiones de un competidor.} es tal que si todas las firmas de $n$ a $n-1$ eligieran su mejor estrategia $a^* \equiv (a_1^*,\ldots,a_N^*)$ el n-ésimo jugador no tendrá incentivos a cambiar de estrategia, por lo que nos encontraríamos en el Equilibrio de Nash.\footnote{Bajo estrategias puras, piedra, papel y tijeras no tiene un equilibrio de Nash.} 

Veremos un modelo de competencia monopolítica donde las firmas fijan precios y otro en donde decidan de manera independiente cuanto producir. En el corto plazo las firmas suelen tomar acciones respecto a la fijación de precios o al nivel de producción, en el largo plazo se podría hablar de entradas a mercado o de inversión en I+D, etc. Los resultados principales para cada modelo serán distintos.\footnote{Algo más apegado a la realidad sería pensar que las firmas en un período elijen cuanto invertir, incidiendo en cuanto podrán producir y luego compiten en precios. Por lo que sería una combinación de ambas.**}

\subsection{Competencia a la Bertrand}

Este modelo fue planteado por el matemático \textbf{Joseph Bertrand}\marginnote{\textbf{Joseph Bertrand (1822-1900):} Matemático francés del siglo XIX. Uno de sus más grandes aportes fue el modelo que lleva su nombre. Fue uno de los críticos del principio de maximización de utilidad.}[0cm] en 1883. Vamos a pensar en un duopolio de firmas $i\in 1,2$ que ofrecen un producto homogéneo compitiendo precios $p_i$. 

Como los productos son sustitutos perfectos, la firma que ofrezca el menor precio se llevará toda la demanda $Q_i(p_i)$, en caso de ofrecer un mismo precio se reparten la demanda de manera equitativa. Por último consideraremos que las firmas tendrán un costo marginal $c_i$ cada una. 

Para entender como se llega al equilibrio en este mercado primero tenemos que identificar cuales son las mejores respuestas de una firma ante acciones de la otra, es decir la función de reacción óptima $a^*_i(a_{-i})$. Pensando desde el punto de vista de la firma $1$ podemos considerar 3 casos posibles y sus respectivas respuestas ante acciones de la firma $2$, buscamos definir la función de reacción $p^*_1(p_2)$.

\begin{itemize}
    \item En caso de que la firma $2$ fije un precio $p_2$ mayor al precio monopólico $p_1^M$. La mejor respuesta es fijar el precio monopólico, de esta manera maximizan beneficios mientras absorben toda la demanda.
    \item Si la firma $2$ fija un precio menor al precio monopólico $p_1^M$ y mayor a al costo marginal $c_1$. Para capturar toda la demanda conviene fijar un precio minúsculamente menor al de la competencia, lo cual se denota como $p_1-\epsilon$ siendo $\epsilon$ un número positivo cercano al cero.
    \item La firma $2$ fija un precio igual o menor al costo marginal $c_1$. Para estos casos la mejor respuesta es fijar el costo marginal.
\end{itemize}

El precio $p_1$ que fije la firma $1$ en función de $p_2$ seguirá la ecuación \ref{eq: funciones de reaccion bertrand} y representado en la figura \ref{fig:funciones de reacción Bertrand},
\begin{align}
    p^*_1(p_2)= \left\{ \begin{array}{lcc} p_1^M & \text{si} &  p_2> p_1^M  \\ \\ p_2-\epsilon & \text{si} & p_1^M \geq p_2>c_1 \\ \\ c_1 & \text{si} & c_1 \geq p_2  \end{array} \right. \label{eq: funciones de reaccion bertrand}
\end{align}

\begin{figure}[htb]
    \centering
    \caption{Funciones de reacción de competencia tipo Bertrand con firmas simétricas}
    \includegraphics[width=15cm]{Figuras/Función de reacción Bertrand.jpeg}
    \label{fig:funciones de reacción Bertrand}
\end{figure}

\textsc{Equilibrio bajo competencia de precios}. Vamos considerar dos casos canónicos de este modelo de competencia por precios simple, uno donde las empresas son igual de eficientes y otro donde una es superior que otra en eficiencia.

Primero tomemos el caso en que las empresas son simétricas, lo cual implica que tienen un mismo costo marginal $c = c_1 = c_2$. La mejor respuesta ante cualquier precio $p_{-i}^M \geq p_i>c$ será fijar un precio menor, ante lo cual la competencia debería responder con un precio aun menor. De esta manera el precio bajará hasta el punto en que tanto $p_1$ como $p_2$ sean iguales a $c$. Este caso es conocido como la \textbf{paradoja de Bertrand}\marginnote{\textbf{Paradoja de Bertrand:} En competencia por precios a la Bertrand bajo ciertas condiciones se llega al mismo precio que en competencia perfecta bajo competencia imperfecta.}[-2cm] pues con apenas dos firmas llegamos a los mismos precios que en competencia perfecta ($p_i=c_i$). Véase la figura \ref{fig:EN Bertrand sim}. 

\begin{figure}[htb]
    \centering
    \caption{Equilibrio de Nash en Bertrand con firmas simétricas}
    \includegraphics[width=15cm]{Figuras/EN Bertrand con firmas simétricas.jpeg}
    \label{fig:EN Bertrand sim}
\end{figure}

Segundo, el caso en que una firma sea más eficiente que la otra, tome por ejemplo que la firma $1$ es más eficiente $c_1<c_2$. Cuando una firma tiene un costo marginal menor a su rival podrá absorber toda la demanda fijando un precio ligeramente menor al costo marginal de su competencia, en este caso sería $p_1 = c_2 - \varepsilon$. Lo cuál llevaría a la firma menos eficiente a salir del mercado y la firma ganadora obtendría beneficios. Véase la figura \ref{fig:EN Bertrand asim}. 

\begin{figure}[htb]
    \centering
    \caption{Equilibrio de Nash en Bertrand con firmas asimétricas}
    \centering
    \includegraphics[width=15cm]{Figuras/EN Bertrand firmas asimétricas.jpeg}
    \label{fig:EN Bertrand asim}
\end{figure}

\subsection{Competencia a la Cournot}

Otra manera de modelar la competencia entre firmas $i \in 1,2$ es considerando que las firmas no fijan el precio sino que este es producto de la cantidad que produzcan en total. En este modelo cada firma produce una cantidad para poder vender, a mayor cantidad produzcan venden más pero también presionan a la baja el precio. Que las firmas decidan la producción en vez del precio aplica bien en mercados con nula diferenciación de producto tales como los commodities.\footnote{Por ejemplo, no importa la marca del cobre, no hay diferenciación de producto. El mayor predictor del precio asumiendo fija la demanda será la oferta.} El matemático francés \textbf{Antoine Cournot}\marginnote{\textbf{Antoine Cournot (1801-1877):} Filósofo y matemático frances que impulso la economía marginalista. Fue de los primeros quienes empezaron a usar funciones matemáticas para describir relaciones como la oferta y la demanda}[-4.5cm] planteó un modelo de mercado de un bien homogéneo donde la única variable estratégica que manejan las firmas es el nivel de producción.

Presentaremos el modelo dentro de un duopolio en donde cada firma produce una cantidad $q_i$, donde el total producido $Q$ es la suma de las producciones individuales.
\begin{equation}
    Q = \sum_{i=1}^N q_i = q_1 + q_2 + \ldots +q_N
\end{equation}
Asumiremos que las firmas son simétricas (mismo costo marginal) y enfrentan una misma demanda inversa lineal.
\begin{equation}
    P(Q) = A - Q \label{eq: demanda inversa lineal}
\end{equation}
Para resolver el modelo debemos de plantear el problema que enfrenta cada firma. Esto es, maximizar beneficios considerando lo que pueden producir y vender $q_i$ y el beneficio marginal de cada producto $P-c$.
\begin{equation}
    \max_{q_i} \quad \Pi _i = (P-c_i)q_i \label{eq: maximización}
\end{equation}
Como todas las producciones de distintas empresas están contenidas en \ref{eq: maximización} mediante el precio,\footnote{De la manera $P = A-\sum_{i = 1}^Nq_i$.} al optimizar obtendremos la cantidad que debiera producir la firma para maximizar sus beneficios en función de las decisiones de su competencia. Esto es, la estrategia óptima $q_i^*$ ante la estrategia del rival $q_j$. Resolvemos para la firma $1$.
\begin{align*}
\max_{q_1} \quad \Pi_1 &= (P-c_1)q_1 = (A-q_1-q_2)q_1 - c_1q_1 \\
 \text{CPO:} \quad \dv{\Pi_1}{q_1} &= A-2q_1 - q_2 -c_1 =0
\end{align*}
Teniendo las condiciones de primer orden solo queda despejar para obtener la función de reacción de la firma $1$ ante la estrategia de la firma $2$.
\begin{equation}
    q^*_1(q_2) = \frac{A-q_2-c_1}{2} \label{eq: Función de reacción 1}
\end{equation}
La mejor estrategia para $1$ dependerá de la producción de $2$. La producción óptima $q^*_1$ bajará en caso de que el rival produzca más $\Delta^+ q_2$, esto ya que aumentar la producción causaría que el precio caiga más de lo ideal por el exceso de oferta. Es por esto que en competencias a la Cournot la pendiente de la función de reacción será negativa.\footnote{En Bertrand hay pendiente positiva en ciertas partes de la función, la intuición es que si el rival pone un precio mayor entonces la firma puedo subir los precios hasta cierto punto y aún así absorber la demanda.}

\textsc{Equilibrio de competencia a la Cournot}. Para llegar a un equilibrio de Nash a la Cournot cada firma internaliza la función de reacción de la otra para llegar a la cantidad estratégicamente óptima para producir. Esto es, reemplazar la función de respuesta $q^*_2(q_1)$ en $q^*_1(q_2)$ o viceversa. 

Tomemos para hacerlo más fácil dos firmas iguales, por lo que $c_1 = c_2 = c$, incluso sabiendo que van a producir lo mismo podemos simplicar $q_1 = q_2 = q$. Dado que para encontrar equilibrios de Nash bajo cournot reemplazabamos la función de reacción de una firma en la otra ahora nos basta con cambiar $q_i = q$. 
\begin{align}
    q &= \frac{A-q-c}{2} \notag \\
    q &= \frac{A-c}{3} \label{eq: producción individual}
\end{align}
Obtenemos la ecuación \ref{eq: producción individual} la cual representa la producción individual de las dos firmas. El precio se determinará por el total de oferta en el mercado, es decir, la producción de cada una de las firmas $Q = 2q$. 
\begin{align}
    P &= A-2 \cdot \left( \frac{A-c}{3} \right) \notag \\
    P &= \frac{A-2c}{3}
\end{align}
Por último podemos determinar los beneficios de cada firma reemplazando los valores en \ref{eq: maximización}
\begin{align}
\Pi_i &= \left(\frac{A-2c}{3}-c \right) \frac{A-c}{3} \notag \\ 
\Pi_i &= \frac{(A-c)^2}{9}
\end{align}

\begin{figure}[htb]
    \centering
    \caption{Equilibrio de Nash en Cournot}
    \centering
    \includegraphics[width=12cm]{Figuras/EN Cournot.jpeg}
    \label{fig:EN Cournot}
\end{figure}

Este ejemplo de equilibrio es con firmas simétricas, en caso de haber una firma más eficiente a esta le convendría producir más. Graficamente la empresa más eficiente debiera tener una función de reacción más desplazada hacia la derecha. 

\textsc{Caso para $n$ firmas}.
Como ejercicio se recomienda hacer este mismo procedimiento para $n$ firmas en el mercado. ¿Qué ocurre cuando $n$ tiende a infinito?

\subsection*{Recapitulación competencia a la Bertrand y Cournot}

Como es un juego estratégico entre firmas, todas mirarán hacia el futuro para formar creencias de lo que hará la otra con tal de responder de la mejor manera. El modelo Bertrand que hemos visto en este caso tiene una solución directa ya sean firmas simétricas o asimétricas. En Cournot es necesario plantear y resolver el problema de maximización de beneficios para poder llegar al equilibrio de Nash. 

Es importante notar que competir por precios es bastante más fuerte que competir por cantidades. De hecho, dos firmas simétricas bajo competencia de precios no tienen beneficio alguno mientras que en cantidades si los tendrán.

Hemos supuesto implícitamente que las empresas tienen capacidad de servir a todo el mercado que quieran. Este supuesto muchas veces no se ajusta a la realidad y se puede levantar dando pasos a otras conclusiones pero que no se desvían mucho de lo ya visto. En el anexo puede encontrar información de restricciones de capacidad en competencia oligopólica. 

\subsection{Competencia a la Stackelberg}

Este modelo fue primero presentado por el economista alemán \textbf{Heinrich von Stackelberg}.\marginnote{\textbf{Heinrich von Stackelberg (1905-1946)}: Economista alemán de ascendencia argentina nacido en Rusia, aportó enormemente a la teoría de juegos. Fue un arrepentido miembro del Partido Nazi y sargento de la SS.}[-1cm] Ya habiendo comprendido el modelo Cournot no es complejo entender el modelo Stackelberg,\footnote{Competencia a la Stackelberg se presenta casi siempre como una competencia fijando cantidades. Sin embargo hay extensiones del modelo a precios, los cuales pueden distar bastante de como se estudia este modelo en este apunte, pero siguen conteniendo la sustancia del modelo que es un juego a turnos. Véase, Stan van Hoesel,
An overview of Stackelberg pricing in networks,
European Journal of Operational Research ó Meng, FL., Zeng, XJ. A Stackelberg game-theoretic approach to optimal real-time pricing for the smart grid. Soft Comput 17, 2365–2380 (2013).} el \textit{twist} con respecto al modelo anterior es que las firmas jugarán por turnos, inevitablemente la primera a jugar tiene la ventaja. 

Supongamos que la firma $1$ es la líder pues juega primero, mientras que la firma $2$ es la seguidora. La firma líder decidirá en el $t = 1$ la cantidad $q_1$ que producirá y en $t=2$ la firma seguidora decidirá su producción $q_2$ en función de $q_1$.

Este es un juego secuencial que se resuelve por inducción. Si miramos el problema desde el final hasta el inicio primero se maximizan los beneficios de la firma $2$, la cual en $t = 2$ ya sabrá qué produjo la firma $1$, de lo cual obtendremos $q_2^*(q_1)$. La firma líder tendrá que maximizar en $t=1$ sujeto a lo que hará la seguidora en $t=2$. Asumamos que ambas firmas tienen iguales costos marginales para simplificar el problema y que compiten por una demanda lineal tal como en la ecuación \ref{eq: demanda inversa lineal}. El problema de la firma $1$ se plantea de la siguiente manera,
\begin{align*}
    \max_{q_1} \quad &\Pi_1 = (P(Q)-c)q_1 \\
    \text{s.a.}\quad &q_2=q_2^*(q_1)
\end{align*}
Dadas las simplificaciones que hicimos podemos considerar la ecuación \ref{eq: Función de reacción 1} como la función de reacción para la firma seguidora. Reemplazamos la restricción en la expresión a maximizar y reescribimos el problema como,
\begin{align*}
    \max_{q_1} \quad &\Pi_1 = \left(A-q_1-c- \left(\frac{A-q_1-c}{2} \right) \right)q_1 \\ & \Pi_1 = \left( \frac{A-q_1-c}{2}   \right)q_1 \\
    \text{CPO:} \quad & \dv{\Pi_1}{q_1} =   \frac{A-c}{2} -q_1 = 0 \\
    & q_1 = \frac{A-c}{2}
\end{align*}
Reemplazando $q_1$ en $q_2^*(q_1)$ obtendremos la producción de la firma seguidora. 
\begin{align*}
    q_2^*(q_1) = \frac{A-(\frac{A-c}{2})-c}{2} \\
    q_2^*(q_1) = \frac{A-c}{4}
\end{align*}
En comparación a Cournot simple, tenemos que la firma líder (seguidora) producirá más (menos) y por tanto conseguir mayores (menores) beneficios. La firma líder tiene más espacio para producir más sin desplomar el precio, en el segundo turno la firma seguidora tendrá que acotar su producción con tal de no bajar más el precio.

\section{Acuerdos colusivos}

Como se puede notar en el problema de la firma, cada empresa maximiza los beneficios propios y no los conjuntos, lo cual es un factor en el nivel competitivo de las firmas y el poder de mercado que ostenten. Sin embargo, la competencia no es buena para las firmas, al haber más firmas lo esperable es que el precio vaya convergiendo al costo marginal, lo cual muestra como van perdiendo poder de mercado. La colusión es una manera en que las firmas se compromoten a colaborar para aumentar las utilidades personales. 

En la realidad se ven distintos tipos de colusiones las cuales dependiendo el producto tomarán diferentes formas; Aumento de precios, reducción de oferta, restricciones territoriales y mecanismos de castigo. Es útil para las instituciones fiscalizadores estudiar y comprender los factores que podrían indicar una colusión o que faciliten un acuerdo colusivo. 

La teoría de juegos nos ayuda a modelar las decisiones de las firmas mediante \textbf{juegos iterativos}.\marginnote{\textbf{Juegos iterativos:} Juegos en los cuales hay más de un período/turno en que se juega. Pueden ser finitos e infinitos.}[-3cm] Sin embargo hay que considerar un punto muy importante, en caso de períodos finitos no es posible el \textbf{equilibrio perfecto en subjuegos}\marginnote{\textbf{Equilibrio perfecto en subjuegos:} En los juegos dinámicos de información perfecta habrá un equilibrio perfecto en subjuegos cuando se de un equilibrio por inducción en donde todas las decisiones sean creíbles.} colusivo, sólo será posible en iteraciones infinitos lo cual se explicará más adelante.

En este tipo de acuerdos habrá un incentivo a traicionar, es decir, desviarse del acuerdo para conseguir incluso mayores utilidades de las que conseguirían coludiéndose. En caso de que esto ocurra las demás empresas seguirán una \textbf{estrategia gatillo}, provocando que se acabe la fase de colaboración y empiece la fase de castigo. El castigo es empezar a competir normalmente por el resto del juego, ya sea a la Bertrand, Cournot, etc.

En los casos finitos de $T$ períodos de tiempo nunca habrá incentivos para cooperar en el último período. Como no hay un mañana $T+1$ no habría fase de cástigo por desviarse en $T$, todos preferiran traicionarse el uno al otro. En $T$ nadie coopera lo cual induce a que nadie coopere en $T-1$, el resultado es que nadie coopera en ninguno de los períodos. Nunca hay colusión con períodos finitos, mientras que existen equilibrios colusivos en $T$ infinitos cumpliendose ciertas condiciones. Una empresa se coludirá perpetuamente en caso de que los beneficios de coludirse sean mayores a los de no hacerlo. 

\subsection*{Planteamiento}

Las firmas al coludirse estarán ponderando si es que los beneficios de traicionar en el corto plazo compensarán los beneficios de mantenerse coludidos en el largo plazo. Si una firma se desvía quiere decir que en ese turno gana los beneficios extraordinarios $\pi^D$, los cuales suelen ser los beneficios monopólicos $\pi^M$, durante los siguientes turnos por estrategia gatillo todos empiezan a competir obteniendo en este caso $\pi^G$.\footnote{$\pi^G$ va a depender del las condiciones del mercado, demanda, competidores, tipo de competencia; Cournot, Bertrand, Stackelberg, etc.}

Además se tiene que considerar que los beneficios de períodos más lejanos al presente tendrán menos peso para las decisiones de las firmas. De manera similar al modelo de consumo intertemporal vamos a ponderar los períodos futuros por un descuento $0 \leq \delta < 1$ para todos los períodos $t \in 0,1,2,\ldots,T$.

\subsubsection*{Condición de colusión para Bertrand}

Vamos a ver un caso específico para introducirnos en la dinámica. En el caso de un duopolio Bertrand con firmas simétricas ($\pi^G=0$ y $\pi^D = \pi^M$) podemos describir los beneficios de desviarse $V^D$ en el período $t=0$ como,\footnote{El factor $\frac{\delta}{1-\delta}$ viene de la suma geométrica.}
\begin{equation}
    V^D = \delta^0 \pi^M + \delta \pi^G + \delta^2 \pi^G+ \ldots + \delta^T \pi^G= \pi^M + \frac{\delta}{1-\delta} \pi^G = \pi^M \label{eq: Desvío del acuerdo competencia a la bertrand}
\end{equation}
\ref{eq: Desvío del acuerdo competencia a la bertrand} denota las utilidades de desviarse y obtener beneficios monopólicos en el primer turno y luego competir a la Bertrand obteniendo cero beneficios. En caso de seguir la colusión calculamos $V^C$ como el reparto de las utilidades monopólicas $\frac{\pi^M}{N}$. 
\begin{equation}
    V^C = \delta^0 \frac{\pi^M}{2} + \delta ^1 \frac{\pi^M}{2} + \delta ^2 \frac{\pi^M}{2} +\ldots + \delta^T \frac{\pi^M}{2} = \frac{\pi^M}{2} + \frac{\delta}{1-\delta} \frac{\pi^M}{2}
\end{equation}
La colusión se dará cuando para cada empresa se cumpla que, 
\begin{equation}
    V^C \geq V^D
\end{equation}
Para evaluar si esto ocurre tendremos que fijarnos que la tasa de descuento $\delta$ cumpla ciertas condiciones,
\begin{align*}
    V^C \geq V^D \Longleftrightarrow \frac{\pi^M}{2} + \frac{\delta}{1-\delta} \frac{\pi^M}{2} \geq \pi^M \\
    \text{Despejando $\delta$ obtenemos la condición} \quad \delta \geq \frac{1}{2}
\end{align*}

Es decir, mientras se cumpla que $\delta \geq \frac{1}{2}$ el acuerdo colusivo se va a dar en todos los períodos. La interpretación es que si las firmas tienen un nivel de paciencia suficientemente alto para valorar los beneficios de coludirse a largo plazo, preferiran mantener el acuerdo antes de los beneficios de corto plazo del desvío. 

\subsubsection*{Colusiones, caso general}

Una fórmula general de expresar lo anterior es la siguiente,
\begin{equation}
    \frac{\delta }{1-\delta} (\pi ^C - \pi^{G}) \geq \pi ^D- \pi^C \label{eq: Condición de colusión generalizada}
\end{equation}

Donde $\pi^C$ serán los beneficios que recibe la empresa al coludirse con las demás,\footnote{Nótese que es diferente a $\pi^M$ puesto que se tienen que repartir entre las firmas coludidas.} $\pi^D$ son los beneficios del turno al desviarse y $\pi^G$ serán los beneficios donde por estrategia gatillo todas las firmas empiecen a competir. \ref{eq: Condición de colusión generalizada} se interpreta como que los beneficios a largo plazo deben ser más valorados que los beneficios a corto plazo. 

De \ref{eq: Condición de colusión generalizada} también podemos despejar el descuento $\delta$. De esta manera conseguiremos el $\delta$ mínimo para asegurar que la colusión se cumpla.
\begin{equation}
    \delta \geq \frac{\pi^D - \pi ^C}{\pi^D- \pi^G} \label{eq: Condición Descuento}
\end{equation}

\subsection{Factores que facilitan o dificultan la colusión}

Hay distintos factores que pueden facilitar la colusión, el factor de descuento es uno de ellos. Derivaremos los siguientes factores:
\begin{itemize}
    \item A mayor cantidad de firmas más díficil es mantener un equilibrio colusivo.
    \item Ante una fase de castigo más severa es sea más fácil mantener la colusión.
    \item En caso de asimetría en las firmas, habrá quienes sean más propensas o menos propensas a mantener el trato.
\end{itemize}

\subsubsection*{Cantidad de firmas y acuerdos colusivos}

Para un número de $N$ firmas iguales compitiendo a la Cournot las utilidades de coludirse bajaran con la cantidad de firmas, $\pi^C = \frac{\pi^M}{N}$. Si en un turno una firma se desvía ganan $\pi^M$, para simplificar diremos que en la fase de castigo $\pi^G = 0$.

Citando \ref{eq: Condición que tiene un nombre del que no me acuerdo ;(} y reemplazando los beneficios para este caso obtenemos el descuento mínimo necesario para que la colusión sea posible $\bar{\delta}$. 
\begin{align*}
    \delta \geq \frac{\pi^M - \frac{\pi^M}{N}}{\pi ^M - 0} \Longleftrightarrow &\bar{\delta} = 1 - \frac{1}{N} \\
    & \frac{\partial \bar{\delta}}{\partial N} > 0
\end{align*}

A mayor cantidad de firmas es más difícil que las empresas se coludan pues requieren ponderar en mayor medida los beneficios a largo plazo del acuerdo. Es por esto que las instituciones fiscalizadores ponen especial atención en mercados con pocas firmas. 

\subsubsection*{Fases de castigo y acuerdos colusivos}

Es intuitivo pensar que si el castigo es mayor es más fácil alcanzar un acuerdo colaborativo. Anteriormente se mencionó como Bertrand es un tipo de competencia más fuerte que Cournot, lo cual conlleva que sea más fácil llegar a acuerdos colusivos en Bertrand que en Cournot. Esto se puede ver directamente en la expresión \ref{eq: Condición que tiene un nombre del que no me acuerdo ;(}, competir a la Cournot suele resultar en que $\pi^G$ sea mayor que en Bertrand. 
\begin{align*}
    \frac{\partial \bar{\delta}}{\partial \pi^G} > 0
\end{align*}

Al haber menos castigo es necesaria mayor paciencia para mantener el acuerdo. 

\subsubsection*{Asimetría en las firmas y acuerdos colusivos}

Un punto muy relacionado a lo anterior es el tema de acuerdos colusivo entre firmas asimétricas. En un acuerdo la firma más eficiente tendrá una suerte de dominancia por sobre la otra puesto que en caso de competir la más eficiente saldrá mejor parada. Es por lo anterior que las firmas más eficientes requerirán de mayor paciencia mínima $\bar{\delta}$ para mantener el acuerdo. 

\begin{appendices}
    \chapter{Anexo}

\section{Cournot con restricciones de capacidad}

De forma complementaria se puede levantar el supuesto de que una sola empresa podría servir a todo el mercado si así lo quisiera. Algo más parecido a la realidad es que una sola firma no puede servir a todo el mercado aunque así lo quisiera. La capacidad de una firma puede denotarse como $K$. Frente a una demanda inversa lineal $P(Q) = A-Q$ y considerando dos firmas idénticas de $c=0$ tendremos restricciones activas de capacidad en caso de que $K \leq A$. 

En caso de que compitan la pregunta es qué parte de la curva de demanda sirve cada firma. Se puede asumir racionamiento eficiente, en caso de haber dos precios distintos el más barato será el primero en venderse a los que están más dispuestos a pagar. Una vez vendida toda la capacidad de la firma más barata la otra firma se queda con el sobrante.
\end{appendices}


\end{document}